\documentclass[a4paper,10pt,oneside]{article}
\usepackage{geometry}
\geometry{a4paper, portrait, margin=1in}
\usepackage{amsmath,amsthm}
 \usepackage{amssymb}
\usepackage{dsfont}                         % Enables double stroke fonts
\usepackage{color}
\usepackage[draft,inline]{fixme}
\usepackage[english]{babel}
\usepackage{graphicx}
\usepackage{float}

\begin{document}
\title{Hand-In Exercise 1: Modeling and Control of Robot Arm}
\author{Name}

\section{System Modeling}
Insert sketch of pendulum including indication of positive direction of angle $\theta$ and where it is zero.
\subsection{Model of One-Degree of Freedom Robot}
Model of system based on physics.
\subsection{Linearized Model of Robot}
Linearized model - both in time-domain and frequency-domain.
\section{Performance Specification}
\subsection{Time-Domain}

\subsection{Frequency-Domain}

\section{Controller Design}

\subsection{Design of PID Controller}
Use the root locus method to design the controller. Show all root locus plots leading to the final design
\subsection{Step Response of Linearized System}
Insert step response showing both output $y$ and control signal $u$.
\section{Simulation}
\subsection{Simulation of Linearized System Model}
Insert simulation results showing both output $y$ and control signal $u$.
\subsection{Simulation of Nonlinear System Model}
Insert simulation results showing both output $y$ and control signal $u$.
%%%%%%%%%%%%%%%%%%%%%%%%%%%%%%%%%%%%%%%%%%%%%%%%%%%%%%%%%%%%%%%%%%%%%%%%%%%%%%%%%
%%%%%%%%%%%%%%%%%%%%%%%%%%%%%%%%%%%%%%%%%%%%%%%%%%%%%%%%%%%%%%%%%%%%%%%%%%%%%%%%%
\end{document}
